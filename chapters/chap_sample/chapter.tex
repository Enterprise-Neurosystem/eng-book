
%
% Important -- Do not update this chapter of subdirectories. 
% This is meant as an example so we can see how to incorporate the various commands in the chapter
% This acts like samples for figures, tables, call-out text etc. 

\chapterauthor{Author Name1}{Affiliation text1}
\chapterauthor{Author Name2}{Affiliation text2}
\chapterauthor{Author Name3}{Affiliation text3}
\chapterauthor{Author Name4}{Affiliation text4}
\chapter{Modeling Topics for Detection and Tracking}
\label{chap:chapter_model}

This is an introductory section. 

\section{Introduction}\label{intro}
The term reliability usually refers to the probability that a
component or system will operate satisfactorily either at any particular  
instant at which it is required or for a certain length of
time. 

Fundamental to quantifying reliability s a knowledge of how to
define, assess and combine probabilities \cite{Bontempi2005Adaptive}. This may hinge on identifying the
form of the variability which is nherent n most processes. If all
components had a fixed known lifetime there would be no need to model
reliability.

A component part for an electronic item is
manufactured at one of three different factories, and then delivered to
the main assembly line.Of the total number supplied, factory A supplies
50\%, factory B 30\%, and factory C 20\%. Of the components
manufactured at factory A, 1\% are faulty and the corresponding
proportions for factories B and C are 4\% and 2\% respectively. A
component is picked at random from the assembly line. What is the
probability that it is faulty? 

\begin{VF}
``A Process is a structured, measured set of activities designed to produce a specific output for a particular customer
or market---A process is thus a specific ordering of work activities across time and space, with a beginning, an end.
and clearly defined inputs and outputs: a structure for action.''

\VA{Thomas Davenport}{Senior Adjutant to the Junior Marketing VP}
\end{VF}

\begin{table}[b!]
%\noautomaticrules
\tabletitle{Now we are engaged $(a_g^a)$ $\big(a_g^a\big)$ in a great civil war, testing whether that
nation, or any nation so conceived.}%
\begin{tabular}{lccc}
\tch{Scene}    &\tch{Reg. fts.} &\tch{Hor. fts.} &\tch{Ver. fts.}\\
Ball &19, 221 &4, 598   &3, 200\\
Pepsi$^a$&46, 281 &6, 898 &5, 400\\
Keybrd$^b$   &27, 290 &2, 968 &3, 405\\
Pepsi    &14, 796 &9, 188 &3, 209\\
\end{tabular}
\end{table}


\subsection{A component part}
A component part for an electronic item is
manufactured at one of three different factories, and then delivered to
the main assembly line.Of the total number supplied, factory A supplies
50\%, factory B 30\%, and factory C 20\%. Of the components
manufactured at factory A, 1\% are faulty and the corresponding
proportions for factories B and C are 4\% and 2\% respectively. A
component is picked at random from the assembly line. What is the
probability that it is faulty \cite{ilyas2004hsn}? 
A component part for an electronic item is
manufactured at one of three different factories, and then delivered to
the main assembly line.Of the total number supplied, factory A supplies
50\%, factory B 30\%, and factory C 20\%. Of the components
manufactured at factory A, 1\% are faulty and the corresponding
proportions for factories B and C are 4\% and 2\% respectively. A
component is picked at random from the assembly line. What is the
probability that it is faulty? 
A component part for an electronic item is
manufactured at one of three different factories, and then delivered to
the main assembly line.Of the total number supplied, factory A supplies
50\%, factory B 30\%, and factory C 20\%. Of the components
manufactured at factory A, 1\% are faulty and the corresponding
proportions for factories B and C are 4\% and 2\% respectively. A
component is picked at random from the assembly line. What is the
probability that it is faulty? 



\paragraph{MultiRelational $k$-Anonymity.} Most works on $k$-anonymity focus on anonymizing a single data table; however, a real-life \cite{diamantaras1996pcn} database usually contains multiple relational tables. This has proposed a privacy model called \emph{MultiR $k$-anonymity} to ensure $k$-anonymity on multiple relational tables. Their model assumes that a relational database contains a person-specific table $PT$ and a set of tables $T_1,\cdots,T_n$, where $PT$ contains a person identifier $Pid$ and some sensitive attributes, and $T_i$, for $1 \leq i \leq n$, contains some foreign keys, some attributes in $QID$, and sensitive attributes. The general privacy notion is to ensure that for each record owner $o$ contained in the join of all tables $PT \Join T_1 \Join \cdots \Join T_n$, there exists at least $k-1$ other record owners share the same $QID$ with $o$. It is important to emphasize that the $k$-anonymization is applied at the \emph{record owner} level, not at the \emph{record} level in traditional $k$-anonymity. This idea is similar to $(X,Y)$-anonymity, where $X=QID$ and $Y=\{Pid\}$.
\begin{extract}
A component part for an electronic item is \cite{hyvarinen2001ica}
manufactured at one of three different factories, and then delivered to
the main assembly line.Of the total number supplied, factory A supplies
50\%, factory B 30\%, and factory C 20\%. Of the components
manufactured at factory A, 1\% are faulty and the corresponding
proportions for factories B and C are 4\% and 2\% respectively. 
\end{extract}
A component part for an electronic item is
manufactured at one of three different factories, and then delivered to
the main assembly line.Of the total number supplied, factory A supplies
50\%, factory B 30\%, and factory C 20\%. Of the components
manufactured at factory A, 1\% are faulty and the corresponding
proportions for factories B and C are 4\% and 2\% respectively. A
component is picked at random from the assembly line. What is the
probability? 

\begin{shadebox}
A component part for an electronic item is
manufactured at one of three different factories, and then delivered to
the main assembly line.Of the total number supplied, factory A supplies
50\%, factory B 30\%, and factory C 20\%. Of the components
manufactured at factory A, 1\% are faulty and the corresponding
proportions for factories B and C are 4\% and 2\% respectively. A
component is picked at random from the assembly line. What is the
probability that it is faulty? 
\end{shadebox}

In most literature on PPDP, they \cite{jolliffe2002pca} consider a more relaxed, yet more practical, notion of privacy protection by assuming limited attacker's background knowledge. Below, the term ``victim" refers to the record owner being linked. We can broadly classify linking models to two families.

\begin{example}
One family considers a privacy threat occurs when an attacker is able to link a record owner to a record in a published data table, to a sensitive attribute in a published data table, or to the published data table itself. We call them \emph{record linkage}, \emph{attribute linkage}, and \emph{table linkage}, respectively. In all types of linkages, we assume that the attacker knows the $QID$ of the victim. 
\end{example}

In record and attribute linkages, we further assume that the attacker knows the presence of the victim's record in the released table, and seeks to identify the victim's record and/or sensitive information from the table \cite{yao2002can}. In table linkage, the attack seeks to determine the present or absent of the victim's record in the released table. A data table is considered to privacy preserved if the table can effectively prevent the attacker from successfully performing these types of linkages on the table \cite{madden2002tta}. Sections~\ref{intro}-\ref{sec:reclinkage} study this family of privacy models.
\begin{equation}
\mbox{var}\widehat{\Delta\Omega\Gamma} = \sum_{j = 1}^t \sum_{k = j+1}^t
\mbox{var}\,(\hat{\alpha}_j - \hat{\alpha}_k)  = \sum_{j = 1}^t
\sum_{k = j+1}^t \sigma^2(1/n_j + 1/n_k). \label{2delvart2}
\end{equation}


An obvious measure of imbalance is just the difference in the
number of times the two treatments are allocated
\begin{equation}
D_n = |n_A - n_B|. \label{2deffD}
\end{equation}
For rules such as deterministic allocation, for which the expected
value of this difference can be calculated, we obtain the population
value ${\cal D}_n$.

\begin{shortbox}
\Boxhead{Box Title Here}
Another family aims at achieving the \emph{uninformative principle}: The published table should provide the attacker with little additional information beyond the background knowledge. There should not be a large difference between the prior and posterior beliefs; otherwise, there is a privacy threat~\cite{jain2004ass, jolliffe2002pca}. Many privacy models in this family are designed for statistical database and do not distinguish attributes in $T$ into $QID$, but some of them could also thwart record, attribute, and table linkages. Section~\ref{intro} studies this family of privacy models.

Let $m$ be a prime number. With the addition and multiplication as 
defined above, $Z_m$ is a field.
\end{shortbox}

Another family aims at achieving the \emph{uninformative principle}: The published table should provide the attacker with little additional information beyond the background knowledge. There should not be a large difference between the prior and posterior beliefs; otherwise, there is a privacy threat~\cite{jain2004ass, jolliffe2002pca}. Many privacy models in this family are designed for statistical database and do not distinguish attributes in $T$ into $QID$, but some of them could also thwart record, attribute, and table linkages. Section~\ref{intro} studies this family of privacy models.

\begin{theorem}\label{1th:Z_m}
Let $m$ be a prime number. With the addition and multiplication as 
defined above, $Z_m$ is a field.
\end{theorem}


\section{Record Linkage Model}\label{sec:reclinkage}

In the privacy attack of \emph{record linkage}, some value $qid$ on $QID$ identifies a small number of records in the released table $T$,
called a \emph{group}. If the victim's $QID$ matches the value
$qid$, the victim is vulnerable to being linked to the small
number of records in the group \cite{madden2005taq}. In this case, the attacker faces
only a small number of possibilities for the victim's record, and
with the help of additional knowledge, there is a chance that the
attacker could uniquely identify the victim's record from the
group.


\begin{table}
\tabcolsep10pt
    \tabletitle{Examples for illustrating attacks}
    \begin{tabular}{|c|c|c|c|}
        \hline
        \textbf{Job} & \textbf{Sex} & \textbf{Age} & \textbf{Disease} \\
        \hline
        Engineer & Male & 35 & Hepatitis \\
        Engineer & Male & 38 & Hepatitis \\
        Lawyer & Male & 38 & HIV \\
        Writer & Female & 30 & Flu \\
        Writer & Female & 30 & HIV \\
        Dancer & Female & 30 & HIV \\
        Dancer & Female & 30 & HIV \\
        \hline
    \end{tabular}
    \label{table:rawpatient}
\end{table}



\subsection{A component part}
A component part for an electronic item is
manufactured at one of three different factories, and then delivered to
the main assembly line.Of the total number supplied, factory A supplies
50\%, factory B 30\%, and factory C 20\%. Of the components
manufactured at factory A, 1\% are faulty and the corresponding
proportions for factories B and C are 4\% and 2\% respectively. A
component is picked at random from the assembly line. What is the
probability that it is faulty? 

\begin{proof}
Most of the proof of this theorem is routine.  It is clear that $0\in Z_m$ 
and $1\in Z_m$ are the 
zero element and identity element. If $a\in Z_m$ and $a\ne 0$, then $m-a$ 
is the additive inverse of $a$. If $a\in Z_m$ and $a\ne 0$, then the 
greatest common divisor of $a$ and $m$ is 1, and hence
there exist integers $s$ and $t$ such that $sa+tm=1$. Thus $sa=1 -tm$ is 
congruent to 1 modulo $m$. Let $s^*$ be the integer in $Z_m$ 
congruent to $s$ 
modulo $m$. Then we also have $s^*a\equiv 1\  \mbox{mod}\ m$. Hence $s^*$ 
is 
the multiplicative inverse of $a$ modulo $m$. Verification of the rest of 
the field properties is now routine.
\end{proof}


Some claim reengineering is too much "top down" in its orientation, planning and execution and, as a result, it fails
to fully consider all elements of the organizational system (such as customers or suppliers). By integrating
reengineering with Whole Systems change, using the Whole Organizational System approach, breakthrough results can be
achieved.

\begin{notelist}{000000}
\notes{Notes:}{The process of integrating reengineering is best accomplished with an engineer, a dog, and a cat.}
\end{notelist}

\begin{figure}
\centerline{\includegraphics[width=350pt, height=200pt]{chapters/chap_sample/figures/cat.eps}}
\caption[List of figure caption goes here]{Figure caption goes here.}
\end{figure}



\begin{figure}[htb]
\centerline{\includegraphics[width=200pt, height=200pt]{chapters/chap_sample/figures/cat.eps}}
\caption[Short figure caption]{Figure caption goes here.
Figure caption goes here.
Figure caption goes here.}
\end{figure}

\begin{figure}
\begin{center}
\subfigure[\label{f8a}]{\includegraphics[angle=90,width=7cm,height=7cm,angle=-90]{chapters/chap_sample/figures/Histogram.eps}}
\subfigure[\label{f8b}]{\includegraphics[angle=90,width=7cm,height=7cm,angle=-90]{chapters/chap_sample/figures/Histogram.eps}}
\end{center}
\caption[The bar charts depict the different risk contributions]{The bar charts depict the different risk contributions (top: 99\% quantile, bottom: 99.9\% quantile) of the business areas of a bank. The black bars
are based on a Var/Covar approach, the white ones correspond to shortfall risk.}
\end{figure}

\subsubsection{H3 A component part }
A component part for an electronic item is
manufactured at one of three \cite{mardia1979ma} different factories, and then delivered to
the main assembly line.Of the total number supplied, factory A supplies
50\%, factory B 30\%, and factory C 20\%. Of the components
manufactured at factory A, 1\% are faulty and the corresponding
proportions for factories B and C are 4\% and 2\% respectively. A
component is picked at random from the assembly line. What is the
probability that it is faulty? 

Some claim reengineering is too much "top down" in its orientation, planning and execution and, as a result, it fails
to fully consider all elements of the organizational system (such as customers or suppliers). By integrating
reengineering with Whole Systems change, using the Whole Organizational System approach, breakthrough results can be
achieved.

A fundamental notion \cite{yao2002can} is that of a\index{subspace}\index{vector 
space!subspace of} subspace of $F^n$. Let $V$ be a nonempty subset of 
$F^n$. Then $V$ is a {\it subspace} of $F^n$ provided $V$ is closed 
under vector addition and scalar multiplication, that is, 
\begin{enumerate} 
\item[\rm (a)] For all $u$ and $v$ in $V$, $u+v$ is 
also in $V$. 
\item[\rm (b)] For all $u$ in $V$ and $c$ in $F$, $cu$ is 
in $V$. 
\end{enumerate} 
Let $u$ be in the subspace $V$. Because $0u=0$, 
it follows that the zero vector is in $V$. Similarly, $-u$ is in $V$ 
for all $u$ in $V$. A simple example of a subspace of $F^n$ is the set 
of all vectors $(0,a_2,\ldots,a_n)$ with first coordinate equal to 0. 
The zero vector itself is a subspace.
\begin{itemize}
\item
It is the organization of work to achieve a result.
\item
It involves multiple steps and coordination of people and information.	
\begin{unnumlist}
\item
What data is available? 
\item
In what form do we have it?
\item
Who has it?
\end{unnumlist}
\item
It establishes management as the enabler and sustainer of process advantage.
\end{itemize}
The tool Process Mapping is used to help management and workers gain a fresh process insight. When applied to the
logistics process, it may help identify areas where breakthrough is possible. Traditional thinking about processes
comes from "process blindness."2 Because managers have not had to pay much attention to processes, this blindness has
plagued most companies for many years. A 1994 study that appeared in Information Week cited that on an operational
level, most senior managers have no idea how their companies operate. In other words, real day-to-day operational
performance is no longer understood, nor is it controlled on a real-time 
basis.\footnote{JIT II is a registered trademark of the BOSE Corporation and was developed by Lance Dixon.}

\begin{definition}\label{1def:linearcomb}
Let $u^{(1)},u^{(2)},\ldots,u^{(m)}$ be vectors in $F^n$, and let 
$c_1,c_2,\ldots,c_m$ be scalars. Then the vector
\[c_1u^{(1)}+c_2u^{(2)}+\cdots+c_mu^{(m)}\]
is called a {\it linear combination} \index{linear combination} of $u^{(1)},u^{(2)},\ldots,u^{(m)}$.
If $V$ is a subspace of $F^n$, then $V$ is closed under vector addition and 
scalar multiplication, and it follows easily by induction that a 
linear combination of vectors in $V$ is also a vector in $V$. Thus 
{\it subspaces 
are closed under linear combinations}; in fact, this can be taken as the 
defining property of subspaces.
The vectors $u^{(1)},u^{(2)},\ldots,u^{(m)}$ {\it span} $V$ \index{spanning set}
(equivalently, form a {\it spanning set} of $V$) provided every vector in 
$V$ 
is a linear combination of $u^{(1)},u^{(2)},\ldots,u^{(m)}$. The zero 
vector can be written as a linear combination of 
$u^{(1)},u^{(2)},\ldots,u^{(m)}$ with all scalars equal to 0; this is a 
{\it trivial linear combination}.\index{linear combination!trivial} The vectors
$u^{(1)},u^{(2)},\ldots,u^{(m)}$ are {\it linearly dependent} provided 
there are scalars $c_1,c_2,\ldots,c_m$, not all of which are zero, such 
that
\[c_1u^{(1)}+c_2u^{(2)}+\cdots+c_mu^{(m)}=0,\]
that is, the zero vector can be written as a {\it nontrivial linear \index{linear combination!nontrivial} 
combination} of $u^{(1)},u^{(2)},\ldots,u^{(m)}$.
\end{definition}


In addition to matrix addition, subtraction, and multiplication, there is 
one additional operation that we define now. It's perhaps the simplest of 
them all. Let $A=[a_{ij}]$ be an $m$ by $n$ matrix and let $c$ be a 
number \cite{hyvarinen2001ica}. Then the matrix $c\cdot A$, or simply $cA$, is the $m$ by $n$ 
matrix obtained by multiplying each entry of $A$ by $c$:
\[c A=[ca_{ij}].\]\index{matrix!scalar multiplication} \index{matrix!scalar multiple of}
The matrix $c A$ is called a {\it scalar multiple} of $A$.
\begin{VT1}

\VH{Think About It...}

Commonly thought of as the first modern computer, ENTAC was built in 1944. It took up more space than an 18-wheeler's
tractor trailer and weighed more than 17 Chevrolet Camaros. It consumed 140,000 watts of electricity while executing
up to 5,000 basic arithmetic operations per second. One of today's popular microprocessors, the 486, is built on a
tiny piece of silicon about the size of a dime.

\VT
With the continual expansion of capabilities, computing power will eventually exceed the capacity for human
comprehension or human control.

\VTA{The Information Revolution}{Business Week}
\end{VT1}
In order to break away from the traditional view where single functions dominate as a natural way of thinking, Keen
and Knapp pose several questions for management to answer:$^3$
\begin{enumerate}
\item
Who exactly is the customer or person to which the outcome.

\item
What must happen for the customer's request to be completely satisfied?
\begin{itemize}
\item
That change is so rapid that an organization needs increased face-to-face communication to make intelligent decisions.
\item
Successful strategies come from envisioning preferred futures.
\item
People will display more commitment to plans they help develop.
\end{itemize}
\item
	Who does the work and how does it come together?
\begin{enumerate}
\item 
That change is so rapid that an organization needs increased face-to-face communication to make intelligent
decisions.

\item
Successful strategies come from envisioning preferred futures.

\item	
People will display more commitment to plans they help develop.
\end{enumerate}
\item
Who does the work and how does it come together?
\end{enumerate}
\section{Glossary}
\begin{Glossary}
\item[360 Degree Review] Performance review that includes feedback from superiors, peers, subordinates, and clients.
\item[Abnormal Variation] Changes in process performance that cannot be accounted for by typical day-to-day variation. Also referred to as
non-random variation.
\item[Acceptable Quality Level (AQL)] The minimum number of parts that must comply with quality standards, usually stated as a percentage.
\item[Activity] The tasks performed to change inputs into outputs.
\item[Adaptable] An adaptable process is designed to maintain effectiveness and efficiency as requirements change. The process is
deemed adaptable when there is agreement among suppliers, owners, and customers that the process will meet
requirements throughout the strategic period.
\end{Glossary}

