\chapter{AI in Presence of Limited Infrastructure}
\label{chap:limited_infra}

\lipsum[1-2]
\section{Introduction}\label{sec:limited_infra:intro}
\lipsum[2-4]
\paragraph{Neccessity is the Mother of Invention}
It is well known that the tight restrictions on poetry act as a forcing function for creativity.
The ecosystem of computing at the Edge is no different; limited bandwidth and limited computational resources force the designer and engineer to rethink a task, to formulate the process with minimal bandwidth, bit-depth, and dimensionality, e.g.~$\mbox{min}\left(\mbox{BBD}\right)$.
Under such constraints, one will likely find that reducing parameters in trained models, reducing the bit-depth and dimensionality of the in-coming raw data stream, and doing so in low-power system-on-chip solutions that are directly attached to the sensor, will elicit parsimonious solutions that will benefit from high generalizability as well as robustnes to concept drift.

\section{Goals}
\subsection{Information Density}
\paragraph{Optimize representations}
\subsection{Optimal Bandwidth Utilization}
\paragraph{Iteration with network measurements in training}
Reshuffle the hyperparameters iteratively 

\section{Benefits}
\subsection{Robustness}